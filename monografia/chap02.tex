\chapter{Modelagem}
\label{cap:modelagem}

\section{Introdução}

%Neste capítulo, alguns modelos são apresentados. para o problema proposto que será a base para os algoritmos introduzidos nos Capítulos 3 e 4. Em seguida, é apresentado como as interferências foram modeladas, como os testes de interferência são definidos e uma propriedade decorrente da modelagem que auxilia na garantia de menores complexidades de tempo.

\section{Modelando Redes usando Grafos}

Para modelar uma rede qualquer usando um grafo $G=(V,E)$, basta fazer as seguintes representações:

\begin{itemize}
\item Os nós da rede são os vértices do grafo, ou seja constituem o conjunto V. Por convenção, $|V|=n$
\item Os enlaces da rede são as arestas do grafo, ou seja constituem o conjunto E. Por convenção, $|E|=m$
\end{itemize}

Para esse trabalho, é importante que as arestas de G sejam direcionadas já que os enlaces especificam a direção da comunicação. A figura abaixo mostra um exemplo de como tal modelagem pode ser feita para uma rede bem simples com $n=5$ e $m=4$.

Modelar uma rede como um grafo é bastante útil pois permite pegar emprestado algumas propriedades úteis da Teoria dos Grafos e fazer diversas manipulações matemáticas. Especificamente, nesse trabalho, a abstração em grafos também ajudou muito na etapa de implementação dos algoritmos.

\section{Árvore de Combinações}

Seja um grafo direcionado $G=(V,E)$ conforme aquele utilizado para modelar uma rede na seção anterior. O conjunto de arestas E representa todos os enlaces ativos para essa rede. É possível definir um subgrafo $C=(V,E)$ de G, tal que $V(C)=V(G)$ e $E(C)' \subseteq E(G)$. Com isso, tem-se que o menor tamanho de $E(C)$ é zero, ou seja, não há enlaces ativos ($E(C)=\emptyset$), e o maior tamanho de $E(C)$ é o próprio $E(G)$ com todos os enlaces possíveis ativos ($E(C)=E(G)$). Entre esses dois extremos, existem $2^m$ combinações de enlaces possíveis, que são subconjuntos de $G$ definidos pela escolha de quais enlaces estão ativos.

Sobre essas combinações de enlaces é possível definir uma árvore de combinações $A=(V,E)$. Os vértices da árvore são todas os subconjuntos possíveis de G. As arestas da árvore ligam qualquer subconjunto em uma altura $h$ a outro em uma altura $h+1$ que pode ser obtido através da adição de um enlace de G. Em $A$, o tamanho de um vértice C a uma altura h na árvore é tal que $m_c=h$.

O número de conjuntos possíveis com tamanho $h=m_c$ é dado pela combinação de $m$ tomados $m_c$ a $m_c$. Nessa árvore, parte-se da combinação vazia $(m_{c}=0)$ como raiz ($h=0$). Os filhos de qualquer vértice $E(C)$ da árvore são obtidos combinando $C$ com um enlace $i \in E$, tal que $i \ni C$. É apresentado na Figura 2.1 um exemplo de árvore de combinações com a estrutura descrita para $m=3$.
%Figura~\ref{fig:arvoreCombinacoes}


%FIGURA Árvore de Combinações para m=3
%\label{fig:arvoreCombinacoes}


No contexto de árvore de combinações, é importante definir também os descendentes e os ancestrais de um vértice. Um descendente de um vértice $u$ em uma árvore é qualquer vértice $v$ que pode ser alcançado a partir de $u$ por algum caminho na árvore. Analogamente, $u$ é ancestral de $v$ se, e somente se, $v$ é descendente de $u$. 

A abstração de árvore de combinações permite que os conjuntos de enlaces sejam percorridos conforme a estrutura da árvore. Com isso é possível desenvolver métodos sistemáticos para buscar e avaliar a viabilidade das combinações de enlaces. Os métodos mais famosos como Busca em Largura e Busca em Profundidade em árvores podem ser considerados nesse caso. Entretanto, sua utilização exige a construção prévia da árvore que apresenta alta complexidade tanto de tempo, quanto de espaço ($O(2^m)$ em ambos os casos).

Os algoritmos que serão apresentados nos próximos capítulos apenas usam a ideia de árvore de combinações em seus projetos. Não há construção da árvore para realizar buscas e isso representa uma diminuição considerável na complexidade de espaço. Além disso, eles fazem uso da propriedade apresentada na próxima seção que por sua vez ajuda a reduzir a complexidade de tempo.

\section{Modelos de Interferência}

No capítulo \ref{cap:introducao}, o teste de interferência foi descrito de maneira intuitiva e pouco detalhada. Nessa seção, eles serão definidos formalmente e uma propriedade decorrente da natureza desses testes será apresentada. Essa propriedade é importante pois permitirá o desenvolvimento de algoritmos de enumeração com melhores desempenhos.

\subsection{Modelo de Interferência Primária}

Existem duas características dos enlaces que limitam a comunicação entre os nós em uma rede sem fio: 

\begin{enumerate}

\item Enlaces {\it half-duplex}: Em canais {\it half-duplex}, os rádios dos dispositivos que compõem a rede não podem transmitir e receber mensagens ao mesmo tempo. Por isso, um nó pode apenas ser ou transmissor ou receptor em um dado cenário.
 
\item Enlaces dedicados: Apesar das mensagens serem transmitidas em várias direções e, portanto, alcançarem diversos nós, elas são direcionadas a um nó específico. 
\end{enumerate}

Ao modelar uma rede em que os enlaces possuam essas duas características, os nós apenas assumem papel de transmissor ou de receptor em, no máximo, um enlace. A figura abaixo mostra algumas redes com e sem essas características.

%Figura 2.2: Exemplos de redes em que (a) a condição I é quebrada; (b) a condição II é quebrada; e (c) as condições I e II são atendidas.

\subsection{Teste de Interferência Primária}

Seja $i$ um enlace de um conjunto $C \subset E$. O nó transmissor e o nó receptor de $i$ são, respectivamente, $s_{i}$ e $r_{i}$. $C$ passa no {\bf Teste de Interferência Primária} (TIP)\abbrev{TIP}{Teste de Interferência Primária} se, e somente se, $\forall i,j \in C, s_{i} \neq s_{j} \& s_{i} \neq r_{j} \& r_{i} \neq s_{j} \& r_{i} \neq r_{j}$. 

\subsubsection{Descrição do Algoritmo}

\begin{algorithm}[h]
	\SetVline
	{\bf input:} Conjunto de enlaces $C$\\
	\ForEach { $i \in C$}{
		\ForEach { $j \in C$, $j \neq i$}{
			\If { (( $s_{i}==s_{j}$ ) $||$ ( $s_{i}==r_{j}$ ) $||$ ( $r_{i}==s_{j}$ ) $||$ ( $r_{i}==r_{j}$ ))}{
				\Return FALSE
			}
		}
	}
	\Return TRUE
\caption{Algoritmo TIP}
\label{alg:tip}
\end{algorithm}

\subsubsection{Prova de Corretude}

O TIP apenas formaliza o que foi definido na subseção anterior. Portanto, o algoritmo está correto. 

\subsubsection{Análise da Complexidade}

Para o pior caso, $|C|=m$, então a complexidade de tempo é $O(m^2)$. A complexidade de espaço é definida pelo maior tamanho de $C$ possível, portanto, $O(m)$.

\subsection{Modelo de Interferência Secundária}

Como mencionado no capítulo \ref{cap:introducao}, o meio de transmissão é compartilhado, então o nó transmissor em um enlace pode interferir nos nós receptores de outros enlaces. Contudo, existe um limite de interferência aceitável que é baseado na SINR ({\it Signal to Interference plus Noise Ratio}) \abbrev{SINR}{Signal to Interference plus Noise Ratio}  dos nós receptores.

A Potência de Recepção $RP(s,r)$ de uma transmissão entre s e r é a potência com que um sinal transmitido por um nó transmissor $s$ a uma potência de transmissão $TP$ é recebido em um nó $r$ seguindo o modelo de propagação. Matematicamente,

\begin{equation}
RP(s,r) = \frac{TP}{(\frac{d_{sr}}{d_{0}})^{\alpha}}
\label{eq:rp}
\end{equation}

onde $\alpha$ é o coeficiente e $d_{0}$ é a distância de referência do modelo de propagação.

Com isso, dado um nó receptor $r_{i}$, consegue-se calcular a potência de recepção em $r_{i}$ em duas situações distintas:

\begin{itemize} %verificar se pode ter lista com 2 itens

\item quando o sinal é transmitido por $s_{i}$, ou seja, é a potência de recepção dentro do próprio enlace $i$; 
\item quando o sinal é transmitido por $s_{j}$, tal que $j \neq i$, ou seja, é a potência de recepção de um sinal transmitido em um outro enlace $j$. Nesse caso, a potência de recepção de tais sinais é chamada de interferência. 

\end{itemize}

Denomina-se Interferência Total $I(i,C)$ a soma das interferências que os nós transmissores de todos os outros enlaces de $C$ exercem sobre o nó receptor do enlace $i$. Ou seja,

\begin{equation}
I(i,C) = \sum_{j \neq i} RP(s_{j},r_{i})
\end{equation}

Finalmente, a $SINR(i,C)$ é a razão entre a potência de recepção em $r_{i}$ referente a transmissão no enlace $i$ e a interferência causada pelo ruído do ambiente $\gamma$ e a interferência total dos outros enlaces no conjunto $C$.

\begin{equation}
SINR(i,C) = \frac{RP(s_{i},r_{i})} {\gamma + I(i,C)}
\label{eq:sinr}  
\end{equation}

Dado um conjunto de enlaces $C$ e tendo calculado $SINR(i,C)$, $\forall i \in C$, compara-se os valores encontrados com uma constante $\beta$, que representa um valor numérico para o limite de interferência tolerado. Se a interferência for muito alta, o valor do denominador na Equação~\ref{eq:sinr} irá aumentar, fazendo o valor da $SINR(i,C)$ diminuir. 

Nesse trabalho, os seguintes valores para as constantes foram usados: $\alpha = 4.0$, $\beta = YYY$ e $\gamma=XXX$. % explicar o porquê desses valores

\subsection{Teste de Interferência Secundária}

No modelo de interferência secundária, $\beta$ é um limite inferior, tal que, se $SINR(i,C) \geq \beta$ , $\forall i \in C$, então $C$ passa no Teste de Interferência Secundária (TIS)\abbrev{TIS}{Teste de Interferência Secundária}.

\subsubsection{Descrição do Algoritmo}

\begin{algorithm}[h]
	\SetVline
	{\bf input:} Conjunto de enlaces $C$\\
	\ForEach { $i \in C$}{
		\If {(( $SINR(i,C)<\beta$ ))}{
			\Return FALSE
		}
	}
	\Return TRUE
\caption{Algoritmo TIS}
\label{alg:tis}
\end{algorithm}

\subsubsection{Prova de Corretude}

O TIS apenas formaliza o que foi definido na subseção 2.2.2. Portanto, está correto.

\subsubsection{Análise da Complexidade}

Como itera-se sobre todos os enlaces de C para calcular $SINR(i,C)$, sua complexidade é $O(m)$. Devido o laço definido na linha 2 iterar sobre no máximo m enlaces, então a complexidade de tempo é $O(m^2)$. Analogamente ao TIP, a complexidade de espaço é $O(m)$.

\subsection{Viabilidade de Conjuntos}

Dados os modelos de interferência apresentados, se um conjunto de enlace $C$ passar em ambos os testes, então diz-se que $C$ é viável. O algoritmo para testar a viabilidade de um conjunto de enlaces é simplesmente a junção dos algoritmos anteriores.

\subsubsection{Descrição do Algoritmo}

\begin{algorithm}[h]
	\SetVline
	{\bf input:} Conjunto de enlaces $C$\\
		\If { (TIP(C)) $\&\&$ (TIS(C))}{
			\Return TRUE
		}
		\Else {
			\Return FALSE
		}
\caption{Algoritmo VIAVEL}
\label{alg:viavel}
\end{algorithm}

\subsubsection{Prova de Corretude}

A condicional da linha 2 garante que ambos os testes são executados e, somente ao passar necessariamente nos dois, um conjunto C é classificado como viável. Portanto, o algoritmo está correto.

\subsubsection{Análise da Complexidade}

O pior caso para o teste de viabilidade é quando o conjunto é viável, ou seja, todas as iterações do TIP e do TIS ocorrem. Para o maior tamanho de C possível, temos que a complexidade de tempo desse algoritmo é $O(m)+O(m^2)=O(m^2)$.

\subsection{Inviabilidade Hereditária}

Dados os modelos de interferência apresentados, se um conjunto de enlace $C$ passar em ambos os testes, então diz-se que $C$ é viável. Entretanto, nesta subseção, será analisado o que acontece quando C não é viável.

Em um primeiro cenário, assume-se que $C$ não passou no TIP. Nesse caso, pelo menos um nó de $C$ está participando de mais de um enlace, o que é proibido. Seja um conjunto $C'$, tal que $C' = C \cup \{i\}$, onde $i \in E$. A adição do novo enlace $i$ em $C$ pode: ({\bf i}) conectar dois nós contidos em $C$; ({\bf ii}) conectar um nó existente em $C$ a um novo nó; e ({\bf iii}) incluir dois novos nós conectados por $i$. 
  
Nas três situações descritas anteriormente, a adição do novo enlace $i$ para formar $C'$ não muda o fato de que $C$ não é viável, independentemente do efeito que $i$ cause no conjunto original $C$. Consequentemente, é possível notar que $C'$ também não é viável.

Em um segundo cenário, assume-se que $C$ passou no TIP, mas não passou no TIS. Nesse caso, $SINR(i,C) < \beta$, para algum enlace $i$. Seja um conjunto $C'$, tal que, $C' = C \cup \{j\}$, onde $j \in E$ e $C'$ também passa no TIP. A adição de um novo enlace ao conjunto $C$ para formar $C'$, apenas fará aumentar a interferência nos enlaces já contidos em $C$. Mesmo que a contribuição na interferência total seja pequena, podendo até ser desprezada, a adição de um novo enlace não muda o fato de que $C$ não é viável. Consequentemente, é possível notar que $C'$ também não é viável.

Os dois cenários apresentados garantem a seguinte propriedade: se um conjunto $C$ não é viável, independentemente de qual teste de interferência ele falhou, então qualquer conjunto $C'$, tal que $C \subset C'$ também não é viável. Usando o modelo de árvore de combinações apresentado na seção anterior, se uma combinação $C$ da árvore de combinações não é viável, então todos os seus descendentes na árvore também não são viáveis. Por isso, essa propriedade é denominada Inviabilidade Hereditária. 

Devido a Inviabilidade Hereditária, no processo de busca e verificação de viabilidade de todas as combinações de enlaces de uma rede, sabe-se que, ao encontrar qualquer combinação não viável, não é necessário testar a viabilidade de nenhum de seus descendentes. O ato de não testar os descendentes de uma combinação não viável pode ser chamado de ``podar'' a árvore. 

\section{Exemplo com Busca em Profundidade}

O grafo G=(V,E) da Figura 2.3 representa os nós e os enlaces de uma rede mesh sem fio. Deseja-se encontrar os conjuntos de enlaces viáveis dessa rede. 

%Figura 2.3: Rede Mapeada no Grafo G=(V,E)

Nesse exemplo, $E=\{1, 2, 3, 4\}$, ou seja, $m=4$. Portanto, existem $2^4 = 16$ combinações de enlaces que são representados na árvore de combinações da Figura 2.4. Uma Busca em Profundidade será executada para percorrer os vértices da árvore que serão verificados usando o algoritmo VIÁVEL. A ordem em que os vértices são visitados é $\{\{\}, \{1\}, \{1,2\}, \{1,2,3\}, \{1,2,3,4\}, \{1,3\}, \{1,3,4\}, \{1,4\}, \{2\}, \{2,3\}, \{2,3,4\}, \{2,4\},$ $\{3\}, \{3,4\}, \{4\}\}$ e pode ser verificada também na Figura 2.4.

%Figura 2.4: Árvore de Combinações dos Enlaces de G

A Tabela 2.1 mostra o resultado dos testes para cada combinação e a Tabela 2.2 mostra os motivos pelos quais as combinações falharam os testes.

%Tabela 2.1

%Tabela 2.2

Finalmente, os conjuntos de enlaces viáveis obtidos  são: $\{\{\},\{1\},\{2\},\{2,4\},\{3\},\{3,4\},\{4\}\}$. 

\section{Vantagens e Desvantagens}

No pior caso, o processo de enumeração descrito na seção anterior percorre todos os $2^m$ vértices da árvore, aplicando VIAVEL em cada conjunto de enlaces. Isso representa uma complexidade de $O(2^mm^2)$. Na prática, as redes possuem uma densidade grande. Por isso, é um exagero assumir que $2^m$ combinações serão testadas, pois certamente haverão muitos conjuntos não viáveis. 

Seja F o conjunto de combinações viáveis obtido por meio de um processo de enumeração como o da seção anterior. Certamente, $|F|$ combinações foram testadas, caso contrário não fariam parte de F. Na maioria dos casos, os conjuntos não viáveis são "podados" de forma que só um primeiro conjunto é testado. Supondo o pior caso, sempre depois de um conjunto viável com maior altura em um ramo da árvore ser testado, encontra-se um conjunto não viável. Nesse caso, o número de combinações testadas é igual a $|F| + |U|$, onde U é o conjunto das primeiras combinações não viáveis testadas depois de uma viável. É possível ver que $|F| \approx |U|$ e, portanto, o número de conjuntos testados é $2|F|$. 

Existe, então, um bom candidato a substituir a porção exponencial da complexidade do algoritmo mencionado. Como $O(|F|)$ conjuntos serão testados (viáveis ou não), a complexidade pode ser alterada para $O(|F|m^2)$. Até o momento, $|F|$ é desconhecido, mas certamente varia em função do tamanho da rede. No Capítulo \ref{cap:resultados}, uma função que represente os valores de $|F|$ usando m e n como parâmetros é aproximada, definindo melhor o valor da complexidade.

Mesmo com uma potencial redução da complexidade de tempo ao utilizar o valor $|F|$, a complexidade de espaço desse algoritmo ainda é exponencial. Isso significa que sua implementação está limitada a pequenos valores de m.

\section{Conclusão}

Nesse capítulo, todo o problema foi modelado usando diferentes abstrações com o intuito de obter um procedimento sistemático para enumerar combinações viáveis. Devido a essa modelagem, a complexidade de tempo encontrada em algoritmos que usam força bruta pode ser modificada para um valor diferente, que será estudado em capítulos futuros. Ainda assim, a complexidade de espaço continua exponencial, inviabilizando sua utilização na a maioria das aplicações.

