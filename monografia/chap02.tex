\chapter{Modelagem}
\label{cap:modelagem}

Neste capítulo, é apresentado como modelar uma rede usando grafos e como modelar as combinações de enlaces em uma árvore de combinações. 

Os modelos de iterferências são apresentados com o intuito de definir o teste de viabilidade e, futuramente, permitir algumas otimizações.

Ao final, é apresentado um exemplo de como enumerar os conjuntos viáveis ingenuamente usando os modelos introduzidos com Busca em Largura. As vantagens e desvantagens dessa abordagem são discutidas.

\section{Modelagem Redes usando Grafos}

Para modelar uma rede qualquer usando um grafo $G=(V,E)$, basta fazer as seguintes representações: Os nós da rede são os vértices do grafo, ou seja constituem o conjunto $V(G)$ e, por convenção, $|V|=n$. Os enlaces da rede são as arestas do grafo, ou seja constituem o conjunto $E(G)$ e, por convenção, $|E|=m$.

Para esse trabalho, é importante que as arestas de G sejam direcionadas já que os enlaces especificam a direção da comunicação. A Figura \ref{fig:grafo} mostra um exemplo de como tal modelagem pode ser feita para uma rede bem simples com $n=5$ e $m=4$.

\begin{figure}[htb]
\centering
\includegraphics[width=1\textwidth]{figs/grafo}
\caption[Modelo da rede (a) usando o grafo (b).]
{Modelo da rede (a) usando o grafo (b).}
\label{fig:grafo}
\end{figure}

Modelar uma rede como um grafo é bastante útil pois permite utilizar algumas propriedades importantes da Teoria dos Grafos e fazer diversas manipulações matemáticas. Especificamente, nesse trabalho, a abstração em grafos também ajudou muito na etapa de implementação dos algoritmos.

\section{Árvore de Combinações}

Seja um grafo direcionado $G=(V,E)$. É possível definir um conjunto $C$, tal que $C \subseteq E(G)$, isto é, $C$ é uma combinação dos enlaces de $E(G)$. Com isso, tem-se que o menor tamanho de $C$ é zero, ou seja, não há enlaces ativos ($C=\emptyset$), e o maior tamanho é o próprio $E(G)$, com todos os enlaces possíveis ativos ($C=E(G)$). Entre esses dois extremos, existem $2^m$ combinações de enlaces possíveis, que são subconjuntos de $E(G)$ definidos pela escolha de quais enlaces estão ativos.

Sobre essas combinações de enlaces é possível definir uma árvore de combinações $A=(V,E)$. Os vértices da árvore são todas as combinações possíveis dos enlaces em $E(G)$. As arestas da árvore ligam qualquer combinação em uma altura $h$ a outra em uma altura $h+1$ que pode ser obtido através da adição de um enlace de $E(G)$. Em $A$, o tamanho de um vértice C a uma altura h na árvore é tal que $|C|=h$.

\begin{figure}[htb]
\centering
\includegraphics[width=0.7\textwidth]{figs/arvore}
\caption[Árvore de Combinações para a rede da Figura \ref{fig:grafo}.]
{Árvore de Combinações para a rede da Figura \ref{fig:grafo}.}
\label{fig:arvore}
\end{figure}

Nessa árvore, a raiz ($h=0$) é a combinação com nenhum enlace, ou seja, $C=\emptyset$ ($(|C|=0)$). Os filhos de qualquer vértice $C$ da árvore são obtidos combinando $C$ com um enlace $i \in E(G)$, tal que $i \notin C$. Para a mesma rede modelada na Figura \ref{fig:grafo}, é ilustrada sua árvore de combinações de enlaces na Figura \ref{fig:arvore}.

No contexto de árvore de combinações, é importante definir também os descendentes de um vértice. Segundo [XXX], se P(r,v) é um caminho da raiz $r$ a um vértice $v$ na árvore, $v$ é descendente de $u$ se, e somente se, $u \in P(r,v)$. 

A abstração de árvore de combinações permite que os conjuntos de enlaces sejam percorridos conforme a estrutura da árvore. Com isso, é possível desenvolver métodos sistemáticos para buscar e avaliar a viabilidade das combinações de enlaces. Os métodos mais famosos como Busca em Largura e Busca em Profundidade em árvores podem ser considerados nesse caso. Entretanto, sua utilização exige a construção prévia da árvore que apresenta alta complexidade tanto de tempo, quanto de espaço ($O(2^m)$ em ambos os casos).

Os algoritmos que serão apresentados nos próximos capítulos apenas usam a ideia de árvore de combinações em seus projetos. Não há construção da árvore para realizar buscas e isso representa uma diminuição considerável na complexidade de espaço. Além disso, eles fazem uso da propriedade apresentada na próxima seção que, por sua vez, ajuda a reduzir a complexidade de tempo.

\section{Modelos de Interferência}

No capítulo \ref{cap:introducao}, os testes de interferências foram descritos de maneira intuitiva e pouco detalhada. Nessa seção, eles serão definidos formalmente e uma propriedade decorrente da natureza desses testes será apresentada. Essa propriedade é importante pois permitirá o desenvolvimento de algoritmos de enumeração com melhores desempenhos.

Em redes sem fio, o meio físico é compartilhado. Isso significa que, apesar das mensagens geradas pelo transmissor serem destinadas um receptor específico, elas acabam atingindo todos os nós com diferentes intensidades.

\subsection{Modelo de Interferência Primária}

Na modelagem de redes sem fio, os enlaces são considerados half-duplex, isto é, os rádios dos dispositivos que compõem a rede não podem transmitir e receber mensagens ao mesmo tempo. Por isso, um nó pode apenas ser transmissor ou receptor em um dado instante de tempo.
 
\item Enlaces dedicados: Apesar das mensagens serem transmitidas em várias direções e, portanto, alcançarem diversos nós, elas são direcionadas a um nó específico. 


Ao modelar uma rede em que os enlaces possuam essas duas características, os nós apenas assumem papel de transmissor ou de receptor em, no máximo, um enlace. A figura \ref{fig:primario} ilustra essas características usando diferentes cenários.

\begin{figure}[htb]
\centering
\includegraphics[width=0.7\textwidth]{figs/primario}
\caption[Cenários com diferentes variações das caraterísticas.]
{Cenários com diferentes variações das caraterísticas.}
\label{fig:primario}
\end{figure}

\subsection{Teste de Interferência Primária}

Seja $i$ um enlace de um conjunto $C \subset E$. O nó transmissor e o nó receptor de $i$ são, respectivamente, $s_{i}$ e $r_{i}$. $C$ passa no Teste de Interferência Primária (TIP)\abbrev{TIP}{Teste de Interferência Primária} se, e somente se, $\forall i,j \in C, s_{i} \neq s_{j} \& s_{i} \neq r_{j} \& r_{i} \neq s_{j} \& r_{i} \neq r_{j}$. 

\subsubsection{Descrição do Algoritmo}

\begin{algorithm}[h]
	\SetVline
	{\bf input:} Conjunto de enlaces $C$\\
	\ForEach { $i \in C$}{
		\ForEach { $j \in C$, $j \neq i$}{
			\If { (( $s_{i}==s_{j}$ ) $||$ ( $s_{i}==r_{j}$ ) $||$ ( $r_{i}==s_{j}$ ) $||$ ( $r_{i}==r_{j}$ ))}{
				\Return FALSE
			}
		}
	}
	\Return TRUE
\caption{Algoritmo TIP}
\label{alg:tip}
\end{algorithm}

\subsubsection{Prova de Corretude}

O TIP apenas formaliza o que foi definido na subseção anterior. Portanto, o algoritmo está correto. 

\subsubsection{Análise da Complexidade}

Para o pior caso, $|C|=m$, então a complexidade de tempo é $O(m^2)$. A complexidade de espaço é definida pelo maior tamanho de $C$ possível, portanto, $O(m)$.

\subsection{Modelo de Interferência Secundária}

Como mencionado no capítulo \ref{cap:introducao}, o meio de transmissão é compartilhado, então o nó transmissor em um enlace pode interferir nos nós receptores de outros enlaces. Contudo, existe um limite de interferência aceitável que é baseado na SINR ({\it Signal to Interference plus Noise Ratio}) \abbrev{SINR}{Signal to Interference plus Noise Ratio}  dos nós receptores.

A Potência de Recepção $RP(s,r)$ de uma transmissão entre s e r é a potência com que um sinal transmitido por um nó transmissor $s$ a uma potência de transmissão $TP$ é recebido em um nó $r$ seguindo o modelo de propagação. Matematicamente,

\begin{equation}
RP(s,r) = \frac{TP}{(\frac{d_{sr}}{d_{0}})^{\alpha}}
\label{eq:rp}
\end{equation}

onde $\alpha$ é o coeficiente e $d_{0}$ é a distância de referência do modelo de propagação.

Com isso, dado um nó receptor $r_{i}$, consegue-se calcular a potência de recepção em $r_{i}$ em duas situações distintas:

\begin{itemize} %verificar se pode ter lista com 2 itens

\item quando o sinal é transmitido por $s_{i}$, ou seja, é a potência de recepção dentro do próprio enlace $i$; 
\item quando o sinal é transmitido por $s_{j}$, tal que $j \neq i$, ou seja, é a potência de recepção de um sinal transmitido em um outro enlace $j$. Nesse caso, a potência de recepção de tais sinais é chamada de interferência. 

\end{itemize}

Denomina-se Interferência Total $I(i,C)$ a soma das interferências que os nós transmissores de todos os outros enlaces de $C$ exercem sobre o nó receptor do enlace $i$. Ou seja,

\begin{equation}
I(i,C) = \sum_{j \neq i} RP(s_{j},r_{i})
\end{equation}

Finalmente, a $SINR(i,C)$ é a razão entre a potência de recepção em $r_{i}$ referente a transmissão no enlace $i$ e a interferência causada pelo ruído do ambiente $\gamma$ e a interferência total dos outros enlaces no conjunto $C$.

\begin{equation}
SINR(i,C) = \frac{RP(s_{i},r_{i})} {\gamma + I(i,C)}
\label{eq:sinr}  
\end{equation}

Dado um conjunto de enlaces $C$ e tendo calculado $SINR(i,C)$, $\forall i \in C$, compara-se os valores encontrados com uma constante $\beta$, que representa um valor numérico para o limite de interferência tolerado. Se a interferência for muito alta, o valor do denominador na Equação~\ref{eq:sinr} irá aumentar, fazendo o valor da $SINR(i,C)$ diminuir. 

\subsection{Teste de Interferência Secundária}

No modelo de interferência secundária, $\beta$ é um limite inferior, tal que, se $SINR(i,C) \geq \beta$ , $\forall i \in C$, então $C$ passa no Teste de Interferência Secundária (TIS)\abbrev{TIS}{Teste de Interferência Secundária}.

\subsubsection{Descrição do Algoritmo}

\begin{algorithm}[h]
	\SetVline
	{\bf input:} Conjunto de enlaces $C$\\
	\ForEach { $i \in C$}{
		\If {(( $SINR(i,C)<\beta$ ))}{
			\Return FALSE
		}
	}
	\Return TRUE
\caption{Algoritmo TIS}
\label{alg:tis}
\end{algorithm}

\subsubsection{Prova de Corretude}

O TIS apenas formaliza o que foi definido na subseção 2.2.2. Portanto, está correto.

\subsubsection{Análise da Complexidade}

Como itera-se sobre todos os enlaces de C para calcular $SINR(i,C)$, sua complexidade é $O(m)$. Devido o laço definido na linha 2 iterar sobre no máximo m enlaces, então a complexidade de tempo é $O(m^2)$. Analogamente ao TIP, a complexidade de espaço é $O(m)$.

\subsection{Viabilidade de Conjuntos}

Dados os modelos de interferência apresentados, se um conjunto de enlace $C$ passar em ambos os testes, então diz-se que $C$ é viável. O algoritmo para testar a viabilidade de um conjunto de enlaces é simplesmente a junção dos algoritmos anteriores.

\subsubsection{Descrição do Algoritmo}

\begin{algorithm}[h]
	\SetVline
	{\bf input:} Conjunto de enlaces $C$\\
		\If { (TIP(C)) $\&\&$ (TIS(C))}{
			\Return TRUE
		}
		\Else {
			\Return FALSE
		}
\caption{Algoritmo VIAVEL}
\label{alg:viavel}
\end{algorithm}

\subsubsection{Prova de Corretude}

A condicional da linha 2 garante que ambos os testes são executados e, somente ao passar necessariamente nos dois, um conjunto C é classificado como viável. Portanto, o algoritmo está correto.

\subsubsection{Análise da Complexidade}

O pior caso para o teste de viabilidade é quando o conjunto é viável, ou seja, todas as iterações do TIP e do TIS ocorrem. Para o maior tamanho de C possível, temos que a complexidade de tempo desse algoritmo é $O(m)+O(m^2)=O(m^2)$.

\subsection{Inviabilidade Hereditária}

Dados os modelos de interferência apresentados, se um conjunto de enlace $C$ passar em ambos os testes, então diz-se que $C$ é viável. Entretanto, nesta subseção, será analisado o que acontece quando C não é viável.

Em um primeiro cenário, assume-se que $C$ não passou no TIP. Nesse caso, pelo menos um nó de $C$ está participando de mais de um enlace, o que é proibido. Seja um conjunto $C'$, tal que $C' = C \cup \{i\}$, onde $i \in E$. A adição do novo enlace $i$ em $C$ pode: ({\bf i}) conectar dois nós contidos em $C$; ({\bf ii}) conectar um nó existente em $C$ a um novo nó; e ({\bf iii}) incluir dois novos nós conectados por $i$. 
  
Nas três situações descritas anteriormente, a adição do novo enlace $i$ para formar $C'$ não muda o fato de que $C$ não é viável, independentemente do efeito que $i$ cause no conjunto original $C$. Consequentemente, é possível notar que $C'$ também não é viável.

Em um segundo cenário, assume-se que $C$ passou no TIP, mas não passou no TIS. Nesse caso, $SINR(i,C) < \beta$, para algum enlace $i$. Seja um conjunto $C'$, tal que, $C' = C \cup \{j\}$, onde $j \in E$ e $C'$ também passa no TIP. A adição de um novo enlace ao conjunto $C$ para formar $C'$, apenas fará aumentar a interferência nos enlaces já contidos em $C$. Mesmo que a contribuição na interferência total seja pequena, podendo até ser desprezada, a adição de um novo enlace não muda o fato de que $C$ não é viável. Consequentemente, é possível notar que $C'$ também não é viável.

Os dois cenários apresentados garantem a seguinte propriedade: se um conjunto $C$ não é viável, independentemente de qual teste de interferência ele falhou, então qualquer conjunto $C'$, tal que $C \subset C'$ também não é viável. Usando o modelo de árvore de combinações apresentado na seção anterior, se uma combinação $C$ da árvore de combinações não é viável, então todos os seus descendentes na árvore também não são viáveis. Por isso, essa propriedade é denominada Inviabilidade Hereditária. 

Devido a Inviabilidade Hereditária, no processo de busca e verificação de viabilidade de todas as combinações de enlaces de uma rede, sabe-se que, ao encontrar qualquer combinação não viável, não é necessário testar a viabilidade de nenhum de seus descendentes. O ato de não testar os descendentes de uma combinação não viável pode ser chamado de ``podar'' a árvore. 

\section{Exemplo com Busca em Profundidade}

A rede modelada na Figura \ref{fig:grafo} será utilizada para construir esse exemplo. O objetivo é encontrar os conjuntos de enlaces viáveis dessa rede. 

Nesse exemplo, os nós foram distribuídos aleatoriamente em uma área quadrada de lado $2000m$. As potências de transmissão foram fixadas em $TP=350mW=24.7712dBm$ para todos os nós. Nesse cenário, $m=4$ enlaces foram definidos, $E(G)=\{A, B, C, D\}$. Uma descrição dos enlaces contendo a distância entre os nós e a potência de recepção pode ser encontrada na Tabela \ref{table:enlaces}. Os valores do coeficiente e da distância de referência do modelo de propagação utilizados foram, respectivamente, $\alpha=4,0$ e $d_0=1,0$m.

\begin{table}[h]
\centering
\caption[Descrição dos Enlaces.]
{Descrição dos Enlaces.}
\label{table:enlaces}
\begin{tabular}{lccc}
\hline
Enlace & Distância ($m$) & Potência de Recepção ($10^{-8}mW$)\\ \hline
A=(4,3)	& $322,583$	& $2,77047$\\
B=(2,1)	& $292,245$	& $4,11274$\\
C=(2,3)	& $192,006$	& $22,0730$\\
D=(5,6)	& $179,399$	& $28,9628$
\end{tabular}
\end{table}

Existem $2^4 = 16$ combinações de enlaces que são representados na árvore de combinações da \ref{fig:bp}. Uma Busca em Profundidade será executada para percorrer os vértices da árvore que serão verificados usando o algoritmo VIÁVEL. A ordem em que os vértices são visitados é $\{\{\}, \{A\}, \{A,B\}, \{A,B,C\}, \{A,B,C,D\}, \{A,B,D\}, \{A,C\}, \{A,C,D\}, \{A,D\},$ $\{B\}, \{B,C\}, \{B,C,D\},$ $\{B,D\}, \{C\}, \{C,D\}, \{D\}\}$ e pode ser verificada também na Figura \ref{fig:bp}. A Tabela \ref{table:resultadoviabilidade} mostra o resultado dos testes para cada combinação. 

\begin{figure}[htb]
\centering
\includegraphics[width=1\textwidth]{figs/bp}
\caption[Busca em Largura na árvore de combinações.]
{Busca em Largura na árvore de combinações.}
\label{fig:bp}
\end{figure}

\begin{table}[h]
\centering
\caption[Resultados dos Testes de Viabilidade.]
{Resultados dos Testes de Viabilidade.}
\label{table:resultadoviabilidade}
\begin{tabular}{lcc}
\hline
Conjunto & Situação\\ \hline
\{\} & Dispensa teste\\
\{A\}	& Dispensa teste\\
\{A,B\}	& Falha TIS\\
\{A,B,C\}	& "Podado"\\
\{A,B,C,D\}	& "Podado"\\
\{A,B,D\}	& "Podado"\\
\{A,C\}	& Falha TIP\\
\{A,C,D\}	& "Podado"\\
\{A,D\}	& Falha TIS\\
\{B\}	& Dispensa teste\\
\{B,C\}	& Falha TIP\\
\{B,C,D\}	& "Podado"\\
\{B,D\}	& OK\\
\{C\}	& Dispensa teste\\
\{C,D\}	& OK\\
\{D\}	& Dispensa teste\\
\end{tabular}
\end{table}

O conjunto vazio e os conjuntos com apenas 1 enlace dispensam o teste de viabilidade. O primeiro conjunto a ser testado de verdade é o conjunto \{A,B\} que é reprovado no TIS e, portanto seus descendentes serão "podados". O conjunto \{A,C\} é reprovado no TIP pois A e C compartilham o nó 3 e, com isso, \{A,C,D\} é "podado". Em seguida, \{A,D\} é testado e reprovado no TIS, mas como é uma folha da árvore, a busca continua normalmente, sem "poda". O conjunto \{B,C\} é testado e reprovado no TIP pois B e C compartilham o nó 2 e, consequentemente, \{B,C,D\} é "podado". Finalmente, os conjuntos \{B,D\} e \{C,D\} são testados e classificados como viáveis, já que passam em ambos os testes.
Conclui-se que os conjuntos de enlaces viáveis obtidos são: $\{\{\},\{A\},\{B\},\{B,D\},\{C\},\{C,D\},\{D\}\}$. 

\begin{table}[H]
\centering
\caption{Cálculos das SINR}
\label{table:sinr}
\begin{tabular}{|c|l|c|c|c|}
\hline
\multicolumn{1}{|l|}{Combinação} & Enlaces($i=s_i,r_i$) & \multicolumn{1}{l|}{$d(s_j,r_i)${[}m{]}} & \multicolumn{1}{l|}{$I(j,i)[10^{-11}$mW$]$} & \multicolumn{1}{l|}{$SINR(i, \{i,j\}$){[}mW{]}} \\ \hline
                                 & C=(4,6)          & 1263,04                              & 11,7883                                  & \cellcolor[HTML]{9AFF99}1.115,22962           \\ \cline{2-5} 
\multirow{-2}{*}{\{C,D\}}        & D=(5,6)          & 1590,92                              & 4,68303                                  & \cellcolor[HTML]{9AFF99}2.282,86418           \\ \hline
                                 & B=(2,1)          & 1744,24                              & 3,24112                                  & \cellcolor[HTML]{9AFF99}365,73588             \\ \cline{2-5} 
\multirow{-2}{*}{\{B,D\}}        & D=(5,6)          & 1590,92                              & 4,68303                                  & \cellcolor[HTML]{9AFF99}2.282,86418           \\ \hline
                                 & A=(4,3)          & 1263,04                              & 11,78830                                 & \cellcolor[HTML]{FFCCC9}\textbf{139,97729}    \\ \cline{2-5} 
\multirow{-2}{*}{\{A,D\}}        & D=(5,6)          & 1327,55                              & 9,65865                                  & \cellcolor[HTML]{9AFF99}1.639,77421           \\ \hline
                                 & A=(4,3)          & 192,006                              & 22073,0                                  & \cellcolor[HTML]{FFCCC9}\textbf{0,12547}      \\ \cline{2-5} 
\multirow{-2}{*}{\{A,B\}}        & B=(2,1)          & 673,016                              & 146,224                                  & \cellcolor[HTML]{FFCCC9}\textbf{26,66666}     \\ \hline
\end{tabular}
\end{table}

A justificativa dos resultados dos TIS realizados são mostrados na Figura \ref{table:sinr}. O valor da tolerância de interferência escolhido foi $\beta=316,228$mW. A coluna das distâncias $d(s_j,r_i)$ mostra a distância entre transmissor de j ao receptor de i. Baseada nessa distância, a interferência total $I(j,i)$ na próxima coluna foi calculada. Na última coluna, o valor das {$SINR(i, \{i,j\}$  calculadas são listadas. Aquelas com valores menores que $\beta$ estão sombreadas de vermelho e justificam a combinação relacionada não ter passado no TIS. Para o cálculo da {$SINR(i, \{i,j\}$, o valor do ruído do ambiente utilizado foi $\gamma=8,004 \times 10^{-11}$ mW.

%No Apêndice A, é apresentada a justificativa para a escolha de todos os parâmetros para o Modelo de Interferência Secundária. Os parâmetros usados para construir esse exemplo também foram utilizados na parte de implementação para gerar as redes usadas nos experimentos. 

\section{Vantagens e Desvantagens}

No pior caso, o processo de enumeração descrito na seção anterior percorre todos os $2^m$ vértices da árvore, aplicando VIAVEL em cada conjunto de enlaces. Isso representa uma complexidade de $O(2^mm^2)$. Na prática, as redes possuem uma densidade grande. Por isso, é um exagero assumir que $2^m$ combinações serão testadas, pois certamente haverão muitos conjuntos não viáveis. 

Seja F o conjunto de combinações viáveis obtido por meio de um processo de enumeração como o da seção anterior. Certamente, $|F|$ combinações foram testadas, caso contrário não fariam parte de F. Na maioria dos casos, os conjuntos não viáveis são "podados" de forma que só um primeiro conjunto é testado. Supondo o pior caso, sempre depois de um conjunto viável com maior altura em um ramo da árvore ser testado, encontra-se um conjunto não viável. Nesse caso, o número de combinações testadas é igual a $|F| + |U|$, onde U é o conjunto das primeiras combinações não viáveis testadas depois de uma viável. É possível ver que $|F| \approx |U|$ e, portanto, o número de conjuntos testados é $2|F|$. 

Existe, então, um bom candidato a substituir a porção exponencial da complexidade do algoritmo mencionado. Como $O(|F|)$ conjuntos serão testados (viáveis ou não), a complexidade pode ser alterada para $O(|F|m^2)$. Até o momento, $|F|$ é desconhecido, mas certamente varia em função do tamanho da rede. No Capítulo \ref{cap:resultados}, uma função que represente os valores de $|F|$ usando m e n como parâmetros é aproximada, definindo melhor o valor da complexidade.

Mesmo com uma potencial redução da complexidade de tempo ao utilizar o valor $|F|$, a complexidade de espaço desse algoritmo ainda é exponencial. Isso significa que sua implementação está limitada a pequenos valores de m.

\section{Conclusão}

Nesse capítulo, todo o problema foi modelado usando diferentes abstrações com o intuito de obter um procedimento sistemático para enumerar combinações viáveis. Devido a essa modelagem, a complexidade de tempo encontrada em algoritmos que usam força bruta pode ser modificada para um valor diferente, que será estudado em capítulos futuros. Ainda assim, a complexidade de espaço continua exponencial, inviabilizando sua utilização na a maioria das aplicações.

